\documentclass[10pt,a4paper,twocolumn]{article}
\usepackage[utf8]{inputenc}
\usepackage[english]{babel}
\usepackage{amsmath}
\usepackage{amsfonts}
\usepackage{amssymb}
\usepackage{hyperref}
\usepackage[sort, numbers]{natbib}
\usepackage{parskip}

\title{Making Anti Evil Maid protection of Qubes OS resistant against shoulder
surfing and video surveillance}
\author{Patrik Hagara}

\begin{document}

\maketitle

\begin{abstract}
  \noindent
  Observing the user during early boot should not be sufficient to defeatt he
  protection offered by Anti Evil Maid. Possible avenues for improvement are
  explored and a solution introduced for both high-risk users, leveraging three
  extra devices to increase multistage evil maid attack resistance, and a much
  simpler scheme for users whose threat model does not include such multistage
  attacks. Detailed time line is put forward for implementing this Google
  Summer of Code proposal.
\end{abstract}


\section{Introduction}

Anti Evil Maid (AEM) is an opt-in feature of the Qubes OS \cite{qubes-homepage}
project aiming to provide resistance against so called evil maid type of
security circumvention attacks, when an attacker has physical access to target
computer.

The current implementation of AEM protection in Qubes OS does not offer any
protection whatsoever against shoulder surfing or video surveillance based
attacks (referred to as "observation attacks" throughout this document), as
both the static trusted boot secret text or image unsealed by the Trusted
Platform Module (TPM) and the disk encryption passphrase are seen in the clear
on the screen or keyboard, respectively. This allows an evil maid attacker to
either modify the boot process and insert a malicious payload while still
showing the right static secret or to trivially decrypt computer storage using
captured passphrase.

This Google Summer of Code (GSoC) project aims to reduce the impact of
observation attacks by changing the way TPM is used for machine to user
authentication from simple visual static secret verification to a method
resistant to such attacks. Similarly, the user to machine authentication is
upgraded from a simple passphrase to two factor authentication (2FA) so that an
attacker knowing only a passphrase cannot compromise the system.

Given the unique security properties and capabilities of TPM devices, multiple
possible avenues for machine to user authentication present themselves. For
instance, the TPM can be used to unseal a secret seed for time based one time
passphrase (TOTP) generator, a six digit output of which the user would verify
on a separate 2FA device (perhaps a closely guarded smartphone with TOTP
application). As for the options of user to machine authentication, TPM-sealed
and passphrase-protected Linux Unified Key Setup (LUKS) key files could be
used. An attacker would then have to both know the passphrase and obtain a copy
of the key file itself, which would be again stored on a small device the user
always carries with them and plugs into the computer only after a verified
trusted boot.

Additionally to the GSoC project idea \cite{gsoc-proj-ideas} written by the
Qubes OS team, this proposal strives to also provide the best possible user
experience (UX) for users whose threat model does not include multistage
software based evil maid attacks.


\section{Project goals}

The main goal of this GSoC project is to substantially reduce the impact of
observation attacks against AEM-enabled Qubes OS installations while not
unnecessarily increasing the user perceived complexity of AEM boot process.
This goal will be achieved by implementing support for and documenting opt-in
machine to user authentication mechanism leveraging Intel Trusted Execution
Technology (TXT) Dynamic Root of Trust Measurement (DRTM) and TPM assisted
measured boot to unseal a LUKS key file stored on removable media.  The LUKS
key file will be protected by an additional passphrase in order to provide user
to machine authentication mechanism. Fallback support must also be developed to
allow recovery in case of LUKS key file loss or unavailability.

Since the primary goal implementation will not be able to fully protect users
whose threat model includes multistage software based evil maid attacks, the
following features listed should also be implemented either during the three
month GSoC work period (should the time permit) or at any point before or after
the event.

Stretch goals are to implement, document and provide fallback options for:
\begin{itemize}
  \item opt-in usage of TOTP tokens for machine authentication
  \item opt-in storage of TPM-sealed and passphrase-protected LUKS key file on
    a secondary AEM media (inserted only after TOTP code verification)
\end{itemize}

Both of the above features trade ease of use for increased resistance against
software based multistage evil maid attacks. However, while not completely
preventing such attacks, they do extend the attack surface by requiring
multiple devices which can be independently copied, altered or seized by an
attacker. Thus, potential users need to carefully evaluate which of the AEM
setup options are most suited for their particular threat model, if any.

Deliverables (applicable to both primary and stretch goals):
\begin{itemize}
  \item AEM packages with listed features implemented
  \item documentation for enrolling and upgrading users
\end{itemize}

Outstanding issues (tracked in qubes-issues GitHub repository
\cite{qubes-issues}) to be resolved by this project:
\begin{itemize}
  \item Option to use AEM secret as LUKS key file \cite{issue-1979}
\end{itemize}


\section{Implementation}

Since Qubes OS aims to be "a reasonably secure operating system" with an
emphasis on being user friendly, the primary goal implementation aims to
deliver the best possible UX for users whose threat model does not include high
enough probability of multistage evil maid attacks. Pros and cons of this
choice were discussed \cite{aem-discussion-thread} on the qubes-devel mailing
list. The workflow shall be as follows:
\begin{enumerate}
  \item user inserts their closely guarded AEM boot media containing tboot
    bootloader, Xen hypervisor, dom0 Linux kernel, initrd and associated
    configuration along with TPM-sealed and passphrase-protected LUKS key file
    and makes computer boot from it
  \item measured boot is performed and then, assuming neither the computer
    firmware nor software has been maliciously modified, the Platform
    Configuration Register (PCR) values inside the TPM will match the state
    needed to unseal the LUKS key file
  \item once the key file is unsealed, user is prompted to enter their
    passphrase in order to decrypt the LUKS key file
  \item assuming the passphrase supplied by user was able to decrypt the key
    file, internal computer drive is unlocked and computer continues booting
    the OS
\end{enumerate}

Workflow of the stretch goal implementation which is able to resist some of the
possible multistage evil maid attacks is slightly more involved:
\begin{enumerate}
  \item user inserts their closely guarded AEM boot media into the computer and
    makes computer boot from it
  \item bootloader performs a measured boot, attempts to unseal TOTP seed using
    TPM and, if successful, displays a derived six digit code
  \item user checks whether displayed TOTP code matches the one computed by
    their 2FA device
  \item if and only if the TOTP codes match, user can safely insert secondary
    AEM storage media containing the TPM-sealed, passphrase-encrypted LUKS key
    file
  \item the key file is automatically unsealed by TPM
  \item user is prompted for key file decryption passphrase
  \item assuming correct passphrase was entered and the key file is able to
    unlock LUKS-protected internal disk, the OS continues booting
\end{enumerate}

In both cases, an attacker observing both screen and keyboard during early boot
cannot mount an evil maid attack without either:
\begin{itemize}
  \item breaking the measured boot via a vulnerability in Intel TXT, TPM or
    another vital part of the trusted boot process
  \item copying or seizing the AEM boot media and the LUKS key file (possibly
    stored on separate media), enabling the attacker to access encrypted
    computer contents upon seizing it
\end{itemize}

Applicable to the stretch goal implementation only, following additional attack
is possible:
\begin{itemize}
  \item gaining access to or modifying the TOTP seed, changing time and/or date
    of the 2FA device or replacing the whole device, allowing exfiltration of
    AEM boot media contents and LUKS key file, thus enabling the attacker to
    seize the computer and access its encrypted contents
\end{itemize}

If using TOTP token for machine authentication, user should take care to first
memorize the six digit code shown on the computer screen and only then take out
their 2FA device to generate a verification code. This avoids a possible attack
whereas an attacker is able to remotely alter computer screen contents in
(near) real time.

Please note that AEM protection must be installed on removable media in all
cases as observation attacks would capture the TPM Storage Root Key (SRK)
passphrase and make it trivial for an attacker to unseal the LUKS key file,
should they come to possess it. Combined with observing the passphrase used for
the key file, the attacker will be able to unlock the encrypted storage of the
target computer upon seizing it (simply copying contents of the internal
computer drive is not enough as possessing the TPM chip is still required for
key file unsealing).


\section{Timeline}

\begin{description}
  \item[May 30 - Jun 6:] research and implement a proof of concept (PoC) for
    generating, enrolling, passphrase-protecting and TPM-sealing a LUKS key
    file
  \item[Jun 6 - Jun 13:] merge the above PoC into AEM installer, rebuild the
    package and test
  \item[Jun 13 - Jun 27:] implement PoC for disk unlocking using a TPM-sealed,
    passphrase-protected LUKS key file; research how to integrate it into
    Dracut initrd and Plymouth; merge the PoC into AEM, rebuild the package and
    test; first evaluation period starts
  \item[Jun 27 - Jul 4:] implement fallback options; merge fallback option
    changes into AEM package, rebuild and test; first evaluation period ends
  \item[Jul 4 - Jul 11:] document enrollment and upgrade procedures; rebuild
    the AEM package to include documentation
  \item[Jul 11 - Jul 18:] engage the community in both reviewing
    documentation/code and testing the implementation; fix any bugs found
  \item[Jul 18 - Jul 24:] implement PoC for generating/importing TOTP seed,
    transferring generated seed into 2FA device, TPM-sealing the seed and
    generating TOTP codes
  \item[Jul 24 - Jul 28:] second evaluation period
  \item[Jul 28 - Aug 4:] implement fallback options for TOTP, displaying and
    refreshing the TOTP code on Plymouth boot screen; merge that into AEM
    package, rebuild, test
  \item[Aug 4 - Aug 11:] implement support for secondary AEM device holding
    LUKS key file; integrate it into the AEM package, rebuild, test
  \item[Aug 11 - Aug 21:] implement fallback options for secondary AEM device;
    merge it into the AEM package, rebuild, test; document enrollment and
    upgrade steps
  \item[Aug 21 onward:] engage the community in both reviewing
    documentation/code and testing the implementation; fix bugs found by the
    community
\end{description}

Although currently having a twenty hours per week internship, I am confident I
will be able to dedicate another thirty to forty hours per week to GSoC, as
that would be roughly equal to my schedule during the school year. However,
should this be deemed a concern, I could suspend my internship for the three
month GSoC working period. No other commitments scheduled during the relevant
time frame.

I will be sending progress updates to the qubes-devel mailing list, plus weekly
formal summaries.


\section{About me}

Second year undergraduate in Information Security at the Faculty of Electrical
Engineering and Communication, Brno University of Technology.  Previously
studied Computer Networks and Communication for two academic years at the
Faculty of Informatics, Masaryk University.

Passionate about free and open-source software, computer security and digital
privacy. Started coding at the age of twelve, presently focusing on Python.
Installed Qubes OS in mid-2015 and began using it as a primary OS shortly
thereafter. 

Almost four years of quality engineering internship at Red Hat. Comfortable
working both independently and as part of a team, large code bases are fine,
too. Management distance of eight time zones has never been an issue, neither
has the fact of not being a native English speaker.

Not submitting GSoC proposal to any other participating organization.

Living in Brno, Czech Republic (UTC+2 / CEST time zone during the GSoC event).

Contact information:
\begin{itemize}
  \item name: Patrik Hagara
  \item e-mail: \href{mailto:patrihagar@gmail.com}{patrihagar@gmail.com}
  \item GPG: \texttt{0x5C1E71DF031F9AE5}
  \item GitHub \cite{my-github} and LinkedIn \cite{my-linkedin}
\end{itemize}


\def\bibfont{\footnotesize}
\begin{thebibliography}{9}
  \bibitem{qubes-homepage}
    \url{https://www.qubes-os.org/}
  \bibitem{gsoc-proj-ideas}
    \url{https://www.qubes-os.org/gsoc/}
  \bibitem{qubes-issues}
    \url{https://github.com/QubesOS/qubes-issues}
  \bibitem{issue-1979}
    \url{https://github.com/QubesOS/qubes-issues/issues/1979}
  \bibitem{aem-discussion-thread}
    \url{https://groups.google.com/d/topic/qubes-devel/0kOgeiTstJ8/discussion}
  \bibitem{my-github}
    \url{https://github.com/phagara}
  \bibitem{my-linkedin}
    \url{https://www.linkedin.com/in/patrikhagara}
\end{thebibliography}

\end{document}
